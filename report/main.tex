%%%%%%%%%%%%%%%%%%%%%%%%%%%%%%%%%%%%%%%%%
% Short Sectioned Assignment
% LaTeX Template
% Version 1.0 (5/5/12)
%
% This template has been downloaded from:
% http://www.LaTeXTemplates.com
%
% Original author:
% Frits Wenneker (http://www.howtotex.com)
%
% License:
% CC BY-NC-SA 3.0 (http://creativecommons.org/licenses/by-nc-sa/3.0/)
%
% Download template:
% Overleaf (https://www.overleaf.com/8746855dtrgkbkbjjhm)
%
%%%%%%%%%%%%%%%%%%%%%%%%%%%%%%%%%%%%%%%%%

%-------------------------------------------------------------------------------
%	PACKAGES AND OTHER DOCUMENT CONFIGURATIONS
%-------------------------------------------------------------------------------

%\documentclass[paper=a4, fontsize=11pt]{scrartcl} % A4 paper and 11pt font size
\documentclass[a4paper]{article}
%\documentclass[11pt]{article}
%\usepackage[margin=1.25in]{geometry}

%\usepackage[options]{nohyperref}  % This makes hyperref commands do nothing without errors
%\usepackage{url}  % This makes \url work
%\usepackage{hyperref}

\usepackage{graphicx}

%\usepackage[T1]{fontenc} % Use 8-bit encoding that has 256 glyphs
\usepackage[utf8]{inputenc}
%\usepackage{fourier} % Use the Adobe Utopia font for the document - comment this line to return to the LaTeX default
\usepackage[english]{babel} % English language/hyphenation
\usepackage{mathtools,amsmath,amsfonts,amsthm} % Math packages
\usepackage{amssymb}

%\usepackage{lipsum} % Used for inserting dummy 'Lorem ipsum' text into the template

\usepackage{sectsty} % Allows customizing section commands
\allsectionsfont{\centering \normalfont\scshape} % Make all sections centered, the default font and small caps

\usepackage{fancyhdr} % Custom headers and footers
%\pagestyle{fancyplain} % Makes all pages in the document conform to the custom headers and footers
%\fancyhead{} % No page header - if you want one, create it in the same way as the footers below
%\fancyfoot[L]{} % Empty left footer
%\fancyfoot[C]{} % Empty center footer
%\fancyfoot[R]{\thepage} % Page numbering for right footer
\renewcommand{\headrulewidth}{0pt} % Remove header underlines
\renewcommand{\footrulewidth}{0pt} % Remove footer underlines
\setlength{\headheight}{13.6pt} % Customize the height of the header

\numberwithin{equation}{section} % Number equations within sections (i.e. 1.1, 1.2, 2.1, 2.2 instead of 1, 2, 3, 4)
\numberwithin{figure}{section} % Number figures within sections (i.e. 1.1, 1.2, 2.1, 2.2 instead of 1, 2, 3, 4)
\numberwithin{table}{section} % Number tables within sections (i.e. 1.1, 1.2, 2.1, 2.2 instead of 1, 2, 3, 4)

%\setlength\parindent{0pt} % Removes all indentation from paragraphs - comment this line for an assignment with lots of text
\usepackage{indentfirst} % Indentation for all paragraphs

% Used for definitions:
\usepackage{amsthm}
\theoremstyle{definition}
\newtheorem{definition}{Definition}[section]

% To write algorithms in pseudocode:
\usepackage{algpseudocode}
\usepackage{algorithm}

% Don't use colon in algorithms lines:
\algrenewcommand\alglinenumber[1]{\footnotesize #1}

% Input/Output instead of Require/Ensure in algorithms pseudocode:
\renewcommand{\algorithmicrequire}{\textbf{Input:}}
\renewcommand{\algorithmicensure}{\textbf{Output:}}

% To put images side by side:
\usepackage{subcaption}

% Avoid long sentences to go out of margines:
\usepackage{microtype}

% Use URL and avoid long urls to go out of margins (hyphens):
\usepackage[hyphens]{url}

% Define multiline to avoid unindented long sentences in algorithms:
% (https://tex.stackexchange.com/questions/314023/how-to-indent-a-long-sentence-in-an-algorithm)
\usepackage{tabularx}
\makeatletter
\newcommand{\multiline}[1]{%
  \begin{tabularx}{\dimexpr\linewidth-\ALG@thistlm}[t]{@{}X@{}}
    #1
  \end{tabularx}
}
\makeatother

%-------------------------------------------------------------------------------
%	TITLE SECTION
%-------------------------------------------------------------------------------

\newcommand{\horrule}[1]{\rule{\linewidth}{#1}} % Create horizontal rule command with 1 argument of height

\title{
\normalfont \normalsize
\textsc{Sapienza University of Rome} \\ [25pt] % Your university, school and/or department name(s)
\horrule{0.5pt} \\[0.4cm] % Thin top horizontal rule
\LARGE Vision and Perception \\ % The assignment title
\large Mask R-CNN \\
\horrule{2pt} \\[0.5cm] % Thick bottom horizontal rule
}

\author{Ivan Bergonzani, Michele Cipriano,\\Ibis Prevedello, Jean-Pierre Richa} % Your name

\date{\normalsize\today} % Today's date or a custom date

\begin{document}
\sloppy % avoid to make words to go out of margin

\maketitle % Print the title

%-------------------------------------------------------------------------------

\section{Introduction}

The aim of the project is to train a model based on Mask R-CNN\cite{he2017maskrcnn}
using an extended version of COCO that includes the dataset created on Labelbox.
The new dataset consists on a bunch of images that shows the Gymnastic activities
of ActivityNet. All the images have been downloaded from Google using a Python
tool called \texttt{google-images-download}. The idea is to have a working model that
will be later used to classify videos that show Gymnastic activities.

The project has been developed in Python and it has been tested using Google
Compute Engine. The final training has been performed at Alcor lab.

Results. Problem of the network. Peaks. How to improve.

%-------------------------------------------------------------------------------

\section{Training}

Mask R-CNN has a set of losses that are used to check the performances of the
classification, the RPN, the regression on the bounding boxes and the instance
segmentation:
\begin{itemize}
    \item \texttt{smooth\_l1\_loss}: the smooth-L1 loss on the classification of the
        objects.
    \item \texttt{rpn\_class\_loss}: the loss on the classification of the
        object contained in the region proposals, they can either be foreground
        if there is an object inside or background otherwise.
    \item \texttt{rpn\_bbox\_loss}: the loss on the bounding box returned by the RPN.
    \item \texttt{mrcnn\_class\_loss}: the loss for the classifier head of Mask
        R-CNN.
    \item \texttt{mrcnn\_bbox\_loss}: the loss for the bounding box refinement
        at the end of the network.
    \item \texttt{mrcnn\_mask\_loss}: the binary cross-entropy loss for the masks.
\end{itemize}

It is possible to see the graphs of these losses using tensorboard once the
training is complete.

%-------------------------------------------------------------------------------

\section{Conclusion}

Conclusion.

%-------------------------------------------------------------------------------

\newpage
\bibliography{bibliography}
\bibliographystyle{ieeetr}

%-------------------------------------------------------------------------------

\end{document}
\grid
\grid
